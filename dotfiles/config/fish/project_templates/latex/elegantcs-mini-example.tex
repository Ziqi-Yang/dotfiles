\documentclass[lang=en,a4paper,bibtex]{elegantcs}
% NOTE lang=cn when using chinese

\title{Title}
\author{Ethan DENG \\ Fudan University \and Dongsheng DENG \\ PA Technology}
\institute{Your Institute}
\version{0.10}
\date{\today} % NOTE \zhtoday when using chinese
%% \addbibresource[location=local]{reference.bib} % FIXME change the content in the reference.bib file
\begin{document}
\maketitle
\begin{abstract}
abstract
\keywords{Keyword1, Keyword2}
\end{abstract}

Predefined code style for all languages supported by \textit{listings}\\
Code from a file:

\textbf{Cpp}:
\lstinputlisting[language=C++]{assets/test.cpp} % NOTE you may need assets/test.cpp to correctly demostrate the function.

Code in latex document:
\textbf{Java}(with caption version):
\begin{lstlisting}[language=Java, caption={Showcase.java}]
class Showcase
{
  public static void main(String[] args)
  {
    System.out.println("hello world!");
  }
}
\end{lstlisting}

Also, the \lstinline[language=Python]|print("inline code")| example.

\textbf{Ascii Art}:
\setmonofont{Quivira}
\lstinputlisting[language=Ascii]{assets/ascii3.txt}
\defaultfont
\lstinputlisting[language=Ascii]{assets/ascii.txt}

%% \nocite{*}
%% \printbibliography[heading=bibintoc, title=\ebibname]
\end{document}
